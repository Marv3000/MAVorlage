\documentclass[12pt,a4paper,toc=bibliographynumbered, toc=listof, english, draft=false]{scrreprt} %draft

%***************Start Paket-Einbindungen******************

\usepackage{amsfonts,amsmath,amsthm}	
\usepackage{makeidx}	%fuer Verwendung des index'
\usepackage[colorlinks=false,pdfborder={0 0 0},plainpages=false]{hyperref}%fuer hyperlinks im pdf-format
%GENAU DANN verwenden, wenn digital veroeffentlicht
\usepackage{pgf} %fuer grafiken
\usepackage[hang,flushmargin]{footmisc} %Manipulation der Footnotes
\usepackage[square,sort,comma,numbers]{natbib} %Literaturverzeichnis mit Vancouver
\usepackage[ngerman,english]{babel}
\usepackage[utf8]{inputenc}
\usepackage[T1]{fontenc}
\usepackage{setspace}%zeilenabstand
\onehalfspacing %1,5-facher zeilenabstand
\usepackage{lmodern}
\usepackage{microtype}%wichtig f�r Blockabsatz
\usepackage{alnumsec}
\usepackage[printonlyused]{acronym}
\usepackage{pdfpages}
\usepackage{anysize}
\usepackage{ulem}
\usepackage{hyperref}
\usepackage{parskip}%Einzug&Absatz
\setlength{\parindent}{1em}%mit Einzug
\usepackage{graphicx} %Grafiken
\usepackage{lscape}
\usepackage{float} %Float-Umgebung
\usepackage{longtable} %Tabellen �ber zwei Seiten
\usepackage{tabularx}
\usepackage[font=small]{caption} %Float ohne Caption
%\usepackage{titlesec} %Manipulation der Headings
\usepackage{tocstyle}
%\usetocstyle{allwithdot}
\usepackage{graphicx}
\usepackage{xcolor}
\usepackage{booktabs}
\usepackage{fancyhdr} 
\usepackage[nonumberlist, acronym, toc, section]{glossaries}
\usepackage{lipsum}
\usepackage{overpic}
%\usepackage{helvet}
\usepackage[paper=a4paper,left=30mm,right=30mm,top=30mm,bottom=30mm]{geometry}
\usepackage{tikz}

%***************Ende Paket-Einbindungen******************

%**************Start Allgemeine Optionen*****************
\makeindex
\makeglossaries
\setlength{\parindent}{0pt} %kein Einzug (nie)
%\renewcommand{\familydefault}{\sfdefault}
%\newcommand{\abs}[1]{\left\vert#1\right\vert}
%\renewcommand*\familydefault{\sfdefault}
%\titlelabel{\thetitle.\quad}

% Überschriften mit Serifen und BF
\setkomafont{sectioning}{\normalfont\normalcolor\bfseries}


%****************Ende Allgemeine Optionen*************



%************** Start Befehle *************
%Aenderung der Nummerierungsart (klein roemisch)
\renewcommand{\labelenumi}{(\roman{enumi})}

%************** Ende Befehle **************
